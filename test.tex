%%%%%%%%%%%%%%%%%%%%%%%%%%%%%%%%%%%%%%%%%
% University Assignment Title Page 
% LaTeX Template
% Version 1.0 (27/12/12)
%
% This template has been downloaded from:
% http://www.LaTeXTemplates.com
%
% Original author:
% WikiBooks (http://en.wikibooks.org/wiki/LaTeX/Title_Creation)
%
% License:
% CC BY-NC-SA 3.0 (http://creativecommons.org/licenses/by-nc-sa/3.0/)
% 
% Instructions for using this template:
% This title page is capable of being compiled as is. This is not useful for 
% including it in another document. To do this, you have two options: 
%
% 1) Copy/paste everything between \begin{document} and \end{document} 
% starting at \begin{titlepage} and paste this into another LaTeX file where you 
% want your title page.
% OR
% 2) Remove everything outside the \begin{titlepage} and \end{titlepage} and 
% move this file to the same directory as the LaTeX file you wish to add it to. 
% Then add \input{./title_page_1.tex} to your LaTeX file where you want your
% title page.
%
%%%%%%%%%%%%%%%%%%%%%%%%%%%%%%%%%%%%%%%%%
%\title{Title page with logo}
%----------------------------------------------------------------------------------------
%	PACKAGES AND OTHER DOCUMENT CONFIGURATIONS
%----------------------------------------------------------------------------------------

\documentclass[12pt]{ctexart}
\bibliographystyle{plain}
\usepackage[english]{babel}
\usepackage[utf8x]{inputenc}
\usepackage{amsmath}
\usepackage{graphicx}
\usepackage[colorinlistoftodos]{todonotes}

\begin{document}

\begin{titlepage}

\newcommand{\HRule}{\rule{\linewidth}{0.5mm}} % Defines a new command for the horizontal lines, change thickness here

\center % Center everything on the page
 
%----------------------------------------------------------------------------------------
%	HEADING SECTIONS
%----------------------------------------------------------------------------------------

\textsc{\LARGE Zhejiang University}\\[1cm] % Name of your university/college
%
%----------------------------------------------------------------------------------------
%	TITLE SECTION
%----------------------------------------------------------------------------------------

\HRule \\[0.4cm]
{ \huge \bfseries 我的Linux工作环境}\\[0.4cm] % Title of your document
\HRule \\[1.5cm]
 
%----------------------------------------------------------------------------------------
%	AUTHOR SECTION
%----------------------------------------------------------------------------------------

\begin{minipage}{0.4\textwidth}
\begin{flushleft} \large
\emph{Author:}\\
王 \textsc{永康} % Your name
\end{flushleft}
\end{minipage}
~
\begin{minipage}{0.4\textwidth}
\begin{flushright} \large
\emph{Student ID:} \\
\textsc{3210102882} % Supervisor's Name
\end{flushright}
\end{minipage}\\[2cm]

% If you don't want a supervisor, uncomment the two lines below and remove the section above
%\Large \emph{Author:}\\
%John \textsc{Smith}\\[3cm] % Your name

%----------------------------------------------------------------------------------------
%	DATE SECTION
%----------------------------------------------------------------------------------------

{\large \today}\\[0.5cm] % Date, change the \today to a set date if you want to be precise

%----------------------------------------------------------------------------------------
%	LOGO SECTION
%----------------------------------------------------------------------------------------

\includegraphics{ZJU.png}\\[1cm] % Include a department/university logo - this will require the graphicx package
 
%----------------------------------------------------------------------------------------

\vfill % Fill the rest of the page with whitespace

\end{titlepage}


\begin{abstract}

\vspace{5ex}

蒙智慧的王何宇老师感召,吾得以粗窥Linux之高效,
故carefully配置My own解药,以应付漫漫作业之考校.

\end{abstract}

\section{我的工作环境}

\vspace{5ex}

\subsection{我的Linux版本}

\vspace{2ex}

Linux发行版本为Ubuntu Linux 22.04,
内核版本Linux 5.15.0-40-generic x86 64\cite{陈莉君2006Linux}

\subsection{我的调整}

\subsubsection{安装的软件}

\vspace{2ex}

\textbf{Microsoft Edge浏览器}

用于同步浏览器插件和书签,直接迁移Windows环境的浏览器体验

\vspace{2ex}

\textbf{Visual Studio Code}

可以同步写代码的插件和配置,使自己能够快速上手

\subsubsection{额外的配置}

\vspace{2ex}

最主要的是配置了fastgithub,让我能够稍微正常地访问Github

其余还有更换壁纸等让自己看得更舒服的配置调整

\clearpage

\section{我的规划}

\vspace{5ex}

\subsection{使用场景}

\vspace{3ex}

在未来的半年中,写论文时我将极有可能用Linux环境工作.
这可以避免Windows上过多的娱乐软件,游戏软件对我意志力的干扰

\subsection{我的分析}

\vspace{3ex}

我的工作环境基本是可以满足我当前的需求的 ------
对于图形界面还有依赖,Windows的操作习惯仍然存在,
主机依然需要是Windows

而对于未来的工作需求,当我对图形界面的依赖逐渐降低,
对shell的操作愈发熟练,
对Linux的使用更加习惯时,
我可能会从VSCode转向更高效的emacs,
把虚拟机迁移到物理机上,换用其他发行版的Linux

\subsection{数据安全}

\vspace{3ex}

我的工作文件根目录通常是设在一个文件夹,
这个文件夹会留存在本地的同时被自动备份到OneDrive上.
对于重要代码我也会上传至github.
此外我还有额外的机械硬盘用于定期物理备份资料,
所以我的工作系统中数据的安全还是颇有保障的.\cite{俞木发2016同步/备份不闹心}

\clearpage

\bibliography{datalib}

\end{document}