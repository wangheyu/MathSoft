%!Tex Program = xelatex
%\documentclass[a4paper]{article}
\documentclass[a4paper]{ctexart}
\usepackage{xltxtra}
%\setmainfont[Mapping=tex-text]{AR PL UMing CN:style=Light}
%\setmainfont[Mapping=tex-text]{AR PL UKai CN:style=Book}
%\setmainfont[Mapping=tex-text]{WenQuanYi Zen Hei:style=Regular}
%\setmainfont[Mapping=tex-text]{WenQuanYi Zen Hei Sharp:style=Regular}
%\setmainfont[Mapping=tex-text]{AR PL KaitiM GB:style=Regular} 
%\setmainfont[Mapping=tex-text]{AR PL SungtiL GB:style=Regular} 
%\setmainfont[Mapping=tex-text]{WenQuanYi Zen Hei Mono:style=Regular} 


\title{第一讲: 应用数学家应该如何使用计算机}
\author{王何宇}
\date{}
\begin{document}
\maketitle
\pagestyle{empty}

\section{引言}
各位同学早上好, 欢迎参加短学期课程: 数学软件. 这里不少同学可能已经认识我了,
我叫王何宇, 是数学科学学院信息与计算科学系的教师. 因为这里大部分同学都是数院偏应用方向的,
未来三年我们会有很密切的接触.

我的联系方式如下, 如果你有正经事要询问, 请用 email, 如果你有急事要联系, 请用手机.
如果你用其他方式包括钉钉联系我, 不保证能回复.   

因为培养方案的调整, 这个学期的人数格外的多, 大家来自三个年级, 水平也差距很大, 不过问题不大,
因为这门课并不考察你的数学基础和天赋.

已经注意到不少同学在安装虚拟机时遇到了不少问题. 这本身就说明我们这门课的必要性和迫切性.
相信大家都用过很长一段时间的电脑或类电脑产品, 如 ipad 等等. 但是如果到目前为止, 你还不知道什么是 Linux,
那么说明你尚未接触到真正的计算机, 你也不可能发挥出计算机真正的力量.

\subsection{课程主要内容}

大家多少都听过一些著名的数学软件, 比如 Matlab 主要用于数值计算, 或者说带误差的工业或科学计算;
Mathematica 主要用于符号计算, 它可以验证你的微积分或代数的作业;而 R 则是一个重要的统计计算工具.
以上这些重要的数学软件, 我们这个短学期中, 统统不学! 作为成熟的软件的一个必备属性, 就是能让有需要的用户,
通过正确的文档资料, 快速掌握它. 在座作为智商达到人类前 \% 1 的精英人士, 如果一个软件让你觉得学习起来很困难,
那么是这个软件的问题, 而不是你的问题. 当然, 如果是你数学没学好, 那是另一回事. 

我们这个短学期, 主要的任务是:

首先, 作为一个未来的应用数学工作者, 我们应该掌握一些必备的计算机技能.
当前应用数学的发展越来越和现实的工业和科学紧密结合, 在这种趋势下, 数学家需要处理的信息和模型也远比古代更加复杂.
从某种意义上说, 即便是纯粹数学家, 如果能掌握合理的计算机技能,
那么也许他也能够获得比通过黑板和粉笔所得更加好的直观和推理. 而作为应用数学家,
则更加应该充分发挥出计算机这一信息处理工具的力量, 来帮助自己更加好的处理信息, 构建并测试模型, 作出预测,
改进现有理论. 当然, 这里有一条明确的原则: 作为应用数学家,
掌握计算机的原动力必须是能提高自己在数学研究方面的能力和效率, 而不要为了秀技巧而使用计算机. 同时, 和数学一样,
任何问题都有一个最好的做法, 我们在使用计算机时, 也应该体现出我们数学工作的风格和特色,
更加优雅地使用这一本来就是数学家发明的机器.

进一步的, 之前提到的数学软件, 事实上确实已经在科研和工业领域, 发挥着巨大的作用. 要使用这些软件,
关键是你要对你构建的数学模型和背后的理论有深刻的理解. 那么连用户都必须具备精深的数学功底, 更别提开发者了.
我们国家在软件领域是有短板的, 这种短板比制造业上的卡脖子可能更加严重. 因为我们一直以来并不重视软件的开发,
认为白嫖国外的现成软件是一个不错的解决方案. 尽管这一现象近年来在国际形势和国内政策的指引下, 已经有所改变,
但积重难返, 国内目前仍然缺乏比较好的社会环境和共识, 能投入足够多的力量去开发我们能自主的关键软件. 而这当中,
数学软件是情况最危急的一个种类. 它不但能严重卡住我们的教育, 科研和工业的底层进展, 而且, 这些软件的开发尤其困难.
因为程序员好找, 懂编程的数学家国内还不是很多. 而在座的信计和强基班级, 你们当中的不少同学,
未来很可能会进入这个领域. 这个一方面是国家对你们期待, 另一方面也可以说是人生的重大机遇.
在一个伟大的文明上升的过程中, 能参与到其中的一个关键的环节, 随时随地都能遇到的. 当然真正要做到这一点,
我们还需要学习大量的数学和计算机理论和技能. 甚至要持续到博士阶段. 但是, 如果你之前对真正的计算机啥也不知道,
只知道玩游戏, 看视频, 上网聊天, 用用 office 软件, 用百度拼凑你的学习报告,
那么把本课作为一个正经接触计算机的开端也不晚.

当然, 既然这是一门一周精通计算机和应用数学, 从此走上人生巅峰的课程, 所以大家也不要有太多不切实际的幻想.
在这一周, 我们不会教授任何具体的编程技巧, 至少不以这个为目的;当然, 更加不会教授任何数学理论. 从本质上说,
这不是一门数学课, 这是一门给数学人准备的非数学课. 

\subsection{为什么学习 Linux}
大家都知道, 现代计算机需要通过操作系统才能实际工作. 而常用的操作系统有 Windows, Mac OS, Linux, Unix 等等.
这里不管上古的 Unix, Windows 和 Mac 的各种系统都是大家熟悉的, 并且伴随大家成长的. 这些操作系统很容易使用,
也很好玩. 它们在日常的各种娱乐和工作中, 非常方便. 但同时也注定了, 它们在设计的时候, 更多考虑的是让更多的人能快速,
愉快地使用, 而不是提高科研工作者的效率. 从根本上说, Windows 和 Mac 系统的目标是把计算机变成家用电器, 或游戏机, 
让老人和孩子都能愉快地使用. Linux 当然更加难用, 难用的原因就是它不愿意将软硬件资源放在一些对科研来说不必要的功能上,
比如酷炫的动画等等. 事实上现代的 Linux 系统也可以像 Windows 和 Mac 一样酷炫, 但是如果你是用它干活, 干正经的活.
那么其实是你自己选择从更直接的层面去掌握计算资源, 从而以更高的效率在科研路上前进. 

这里我们提出几个重要的原则:

{\bf 人不要去做机器该做的事.}

比如你想知道你的硬盘上有多少学习资料, 有多少已经学习过了, 可以删除, 以清空足够的空间来安装虚拟机,
那么即便在 Windows 下, 你也不应该用鼠标一个目录一个目录去找, 再一个文件一个文件去删.
这样早晚你会痛失一些宝贵的绝版的学习资料. 或者更危险的, 你鼠标右键点击的是删除, 但因为你学习太辛苦了,
不小心点了一个发送, 从而暴露了你努力学习不为人知的事迹. 在这种情况下, 你应该充分发挥计算机提供的搜索和批处理功能,
避免这种重复操作所引起的错误. 当然, 也能由此犯下一个批量的错误. 所以必须熟练才行.

懒惰, 是数学前进的原动力之一. 懒惰导致我们提升抽象的层次, 从而能够用一致的理论解决更多的问题. 而在使用计算机的时候,
也不能太勤劳. 比如前几年(我不知道这几年是不是还是这样), 银行为了避免电子帐目出错, 所有的记账工作, 都要手做一遍...
这种是典型的使用了计算机导致工作效率下降的例子, 大家一定要避免.

{\bf 一切从效率出发, 不要不切实际的功能和约束.}

苹果的操作系统都很漂亮, 也很好用, 同时也很封闭. 封闭提供了安全性, 因此一个苹果的设备, 一般非常安全, 数据不会丢失,
很难被黑客入侵, 也很难中毒. 但这种封闭同时带来了科研工作上的困扰. 因为一个计算机系统,
很难区别一个涉及计算机底层的指令, 是一个科学家在构建一个底层模型, 还是一个黑客在探索漏洞.
这种对科研来说的过保护, 会严重影响工作效率. 所以尽管苹果系统也是一个体系完整的类 Unix 系统,
但是我并不推荐大家直接拿它做数学科研工具. 我自己尝试过, 很快就放弃了.
当然如果你愿意和苹果的各种数字签名和安全策略斗智斗勇, 我也无所谓. 只是我个人觉得, 聪明才智不应该放在这里,
有这个时间多刷点 B 站不香么? 这里顺便说明一下,
我没有任何使用 M1 芯片的经验, 所以你的电脑是使用 M1 芯片的 Mac, 抱歉我不能提供任何技术支持.
我估计我们的助教也做不到. 实际上, 我不认为 M1 芯片的 Mac 是一台适合用于学习的电脑. 

Linux 操作系统和其所属社区采用了另一个方法来加强其安全性, 就是开源. 理论上说, Linux 的每一行代码,
都经过了程序员社区的反复同行评议, 并且共享给所有人. 这个做法, 其实和科研社区的传统是一致的.
我们的每一个科学发现, 也都需要经过同领域同行的评议, 然后才能被广泛接受和传播. 这种模式目前已经成为一种社区文化,
并且支持这个系统永远不会因为某个公司的原因, 引入一切奇怪的特性.

最后, 大家可能已经注意到, 我们培养方案一直在调整. 这既反映国家层面对我们的要求和期望,
同时也是我们学院近年来迅速发展和提升的体现. 对计算, 部分应用和强基的同学,
未来很多专业课程会紧密结合目前的工业应用和科学发展. 因此掌握 Linux 环境将成为一个必须的技能.

\subsection{课程的学习方式}

由于时间所限, 我不会手把手的教大家怎么使用 Linux 系统和其下的软件. 我只会提供要求, 给出方向,
有时有资料, 有时只有一些线索. 然后要靠大家自己去学习和练习. 每天上午, 我们集中在一起,
我给大家介绍我们需要掌握的内容, 给出一些学习建议. 每天下午, 大家可以去机房(不必刷卡), 机房有助教答疑.
也可以在寝室自学, 并完成作业. (其中有两天因为三位一体考试之类的原因, 机房不能使用, 答疑改在教室).

从某种意义而言, 本课完全不存在难度. 因为所谓难, 是指有一个概念, 你无法理解和掌握.
比如说一般人可能无法正确理解什么是无穷小, 而大家也要通过学习和思考才能正确掌握这一概念. 这种学习,
需要一定的天赋和兴趣, 很幸运, 大家都有. 但是我们这门课的计算机内容, 更本不涉及天赋和兴趣. 或者说,
和数学学习相比, 它要求的天赋和智商可以忽略不计. 真正需要的是反复的练习. 因为这些技能,
往往繁琐且不成体系, 每个人都需要通过一定的练习才能正确掌握. 并且还需要在实际工作中反复改进和学习.
比如, 大家在群里的问题, 也并不是每一个我都知道答案的, 因为很多我自己也没遇到过.

大家一定要明确, 掌握一门技能是必须练习的. 隔壁张庆海老师一直在举一个例子, 说是天龙八部中的王语焉,
各种武功理论熟练的不得了, 碰上流氓还是跑不了. 甚至段誉连内力都有了, 还是随便被人捏扁搓圆.
想打死丁春秋, 还得从黑虎掏心开始一招招练. 我们这个课就是学黑虎掏心的, 大家数学很好, 就好比内力已经有了, 但招式也必须练起来.

最后说一下考试, 我们这门课没有考试. 但每天都有一个作业, 一共 7 个作业.
每个作业都是你下午的上机任务. 你至少需要完成其中 5 个作业, 如果完成数超过 5 个,
则取成绩高的 5 个记分. 每个作业 10 分. 每个作业的 DL 都是当天下午的 18:00.
超过这个 DL 提交作业, 成绩会有非线性下降. 7 月 2 日下午 18:00 是全部作业的 DL,
超过这个时间, 不再接受作业. 7 月 2 日下午 18:01 会发布最终项目作业, DL
为当天晚 23:59, 超过时间不再接受提交. 最终项目作业 50 分.
全部成绩之和为你该课程的总评成绩.

\section{进入计算机世界}

我们都知道, 计算机是二进制的. 但当我们玩计算机或电子产品的时候, 呈现在我们面前的计算机是多姿多彩的多媒体形式.
也即文字, 图片, 声音, 电影, 游戏. 然后在计算机文化之类的课中, 我们还学习了很多二进制是如何构建这些媒体的原理,
比如我们知道每 8 位二进制可以构成一个编码, 再对应一个拉丁字母, 这样就是 ASCII 码. 而中文就需要复杂一点,
图片什么的再复杂一些. 大家在学习这些课的时候反应都和现在一样, 啊对对对... 那么有没有同学想过,
要不直接读写二进制? 似乎这个就是最硬核的做法了? 我相信有同学能做到这一点, 比如学过汇编语言的同学.

\subsection{文件的二进制构成}
我们不打算讲汇编语言. 但是能否以一个高效率的方式, 了解一些最简单的媒体的二进制构成? 我们知道,
硬盘上的数据都由文件构成, 那些给人类读的文件, 就是文本文件, 它们也是二进制流, 但是由文字编码构成.
而其他文件, 比如图片, 就完全是二进制数据流. 人类只能通过软件, 间接使用它们.
就没有什么方便的方法直接控制它们么? 否则随着计算机越来越复杂, 你真的觉得是你在使用计算机,
而不是计算机在玩弄你么? 作为我们探索计算机内部的第一步, 让我们先看看是否有优雅的方式,
做一些初步的探索.

首先我们需要有一台电脑, 并且安装了专业的操作系统. Linux 就是一个很好的选择. Linux 有很多不同的发行版,
其中 Ubuntu 是在桌面级别最受欢迎的. 因为它也和 Windows 或苹果系统一样漂亮和方便.
但它确实是一个完整的 Linux 系统. 让我们抛开繁华, 直接按 Ctrl-Alt-T, 打开一个叫终端的上古界面.
这个朴实无华的界面, 让人想起冷战期间的电脑, 在终结者或黑客帝国中到处都是. 确切地讲, 终端是一个设备的概念,
它的原意是用户终端, 或者说给用户使用的设备. 在冷战时代, 很多大型计算机体积非常庞大,
而用户需要使用电话线或其它线缆, 和它通讯. 所以本质上它就是一台能发送和接收文字的电子打印机. 而那个时期的人类,
已经失去了直接和电脑用二进制对话的能力, 所以在终端之上, 还需要运行一个被称为 shell 的程序,
它的任务是在使用字符的人类和使用二进制的计算机内核之间, 建立一个指令翻译和回应环境.
这个环境因为只需要简单的字符指令就可以指挥计算机从事复杂的工作, 比如世界核平, 效率非常高, 所以一直保存到了现在.
目前的情况是, 一方面个人计算机也变得极其复杂, 真正想掌握计算机底层的用户,
仍然需要一个简单直接高效的界面去直接控制计算机. 另一方面, 随着互联网的兴起, 远程控制一台机器也是一个日常需求.
所以几乎所有的操作系统, 都保留了这一层的控制接口. 并且在今天, 如果你真正能熟练使用终端和 shell,
理论上你不但可以控制眼前的机器, 你实际上存在控制和你物理连接的全部设备的可能性, 比如程控交换机, 路由器,
打印机, 手机, 手表... 我知道你现在的想法是 "我有一个不成熟的想法"... 不! 你没有! 我们这门课不涉及这些内容,
你真的像学这些, 请先购买一本最新版的刑法, 然后你可以考虑选修一些安全类的课, 比如白洪欢的课程.

我们还是实际一点, 直接进入一个终端, 看一张图片. 在终端下, 每一个指令都是通过键盘发出的. 对初学者来说,
可能要记忆大量的指令. 但等你熟练之后, 你会发现键盘指令的效率远高于鼠标. 同时也对身体更好.
我们暂时别管编辑软家怎么用, 直接将其转成二进制格式. 除了直接打指令, 还有一种用键盘发送命令的方式是键盘宏,
其实就是大家熟悉的快捷键. 比如 ubuntu 并没有提供窗口最大化的快捷键, 我自己定义一个(允许用鼠标, 尽量少用,
但别迂腐). 面对这一堆二进制我们该干嘛? 别楞着, 这不是数学题, 发呆不会有任何改进. 你必须做点啥, 比如,
不要用百度, 建议用 bing 进行搜索. OK, 你现在知道怎么读这堆数据了. 你可以验证一下. 

所以我们看到, 即便面对二进制流, 我们也不是全无希望的. 因为所有的二进制文件, 都是按照一定的规则编制的.
只要我们掌握了这个规则, 我们同样可以直接切入二进制世界. 但前提是, 首先, 获得这个规则. 这个好办,
一般标准文件, 都是某种工业规范, 只要善用搜索引擎, 不难获得;其次, 具备理解规则, 解析规则, 重构规则的能力.
这往往就是简单的数学和编程能力. 这个大家也不难通过一些快速的学习获得. 不过从刚才的过程中我们不难看到,
作为一切的开始, 我们需要一个终端, 一个字符型的 shell, 以及一个能够编辑文件的编辑软件.
这种编辑软件在界面上不必很花哨, 但是必须能高效地处理字符流, 能快速用键盘发出指令, 同时还要尽可能占用少量的资源.

\subsection{编辑软件}
现在你必须为自己选择一款编辑软件. 它必须有这样的特性: 首先, 这个软件必须能在终端和 shell 的模式下良好运行.
因为有可能你的工作环境是远程的, 在这种情况下, 你唯一拥有的界面就是终端. 不存在使用图形界面的条件.
这就是为什么很多编辑软件看上去极其简陋的原因. 它不愿意将资源消耗在无意义的美化上. 除非这种美化有现实意义.
比如色盲和色弱在数学领域的发病率比较高, 因此我采用了这样一个高对比色的颜色配置. 同时,
高对比色的颜色配置能让我看清更多细节.

这里我使用的编辑软件是 emacs, 它的优点是扩展性强, 强大的宏支持. 但缺点是对资源的占用有点高, 配置较为复杂.
这是一款非常个性的编辑器, 爱的人深爱, 恨的人痛恨.

另一款推荐的编辑软件是 vim, 它的优点是几乎和 emacs 有同样的扩展性和功能, 但对资源占用极低, 适合最基础的通讯条件.
请注意由于默认快捷键的区别, vim 和 emacs 在使用习惯上严重冲突, 导致两个社区之间有严重的政治斗争. 因此,
如果你发现你的研究生导师用的是 emacs, 建议你最好不要用 vim 去刺激他.

第三款捏着鼻子推荐的是 vscode. 这是微软开发的开源软件. 作为开源社区的最大污染源, 微软难得作出了一项正面的贡献.
它确实是为程序员开发的, 而且也可以充分定制. 它的缺点和微软的所有软件一样, 非常消耗资源, 而且文档混乱, 充满误导.

在我们这个工作层面, 编辑器几乎是最重要的工具软件. 给你自己选择一款能将工作效率发挥到最佳的编辑器.
如果你不知道怎么选, 建议你盲从某位大佬. 比如张庆海, 白洪欢, 或者黄文翀.
(忘了白洪欢不用编辑器, 他的母语是二进制的)

不论是那一款编辑器, 对新手而言, 配置都是一个困难的工作. 我个人的建议, 你可以先用缺省配置.
因为缺省配置往往是最多人的选择. 或者, 你直接从哪个大佬那里复制一个配置. 在 Linux 下, 一切都是文件,
特别是一个软件的配置, 其实就是一个配置文件. 同样的文件, 对应同样的配置. 比如 emacs 的配置文件,
默认在你的根目录下一个名为 .emacs 的文件, 前置点导致这个文件是不可见的, 你必须用 ls -al 才能看到它.
如果它确实不存在, 你也可以自己创建它, 或者直接把你复制来的配置文件覆盖上去.

\section{科技文的写作}

作为一个数学工作者, 我们撰写的专业文档, 经常需要有数学公式. 一般来说, 我们对于排版中的数学公式的引入,
需要有如下要求:
\begin{enumerate}
\item 美观, 大方, 和正文和谐一致;
\item 能用键盘输入, 能够统一查找替换, 输入时不影响思考;
\item 除了科技文写作, 也能适应各种商业排版要求, 如图文混排, 海报, 各种杂志等等;
\end{enumerate}

以上要求曾经是科技文写作的一个痛点, 甚至已经严重影响了科研的进展. 比如上个世纪末期,
国内排版一篇科技论文是非常麻烦的. 当然国内还有一个严重的问题是如何中文化.

\subsection{科技排版软件 \LaTeX}
对这个问题的一致性解决办法是由计算机之神: Donald Knuth 给出的一整套解决方案, 从字体到排版.
他是为了撰写他的计算机圣经: The Art of Computer Programming 时遇到这个问题,
并决定彻底解决这个问题. 他给出的排版软件就是 \TeX. 由于 \TeX 太接近计算机底层,
于是科学社区在 \TeX 的基础上提供了 \LaTeX 扩展包. 国内社区又在此基础上给出了 CTEX 扩展包,
最终一个结合本地和国际化, 同时可以直接使用本地字体的版本, 就是我们推荐大家使用的 xelatex 软件.

所有的 \LaTeX 工具, 本质上都很像一段计算机语言, 你必须以源代码的形式, 撰写你的文档,
然后用对应的工具编译, 形成 pdf 或其他版本的输出文件. 比如大家现在看到的 slide,
其实就是一段 latex 代码. 这种专门为了制作 slide 的扩展包, 被称为 beamer.

让我们直接跳到如何学习撰写一篇包含数学公式的文章. 

和学习任何计算机技能一样, 你首先必须收集一些文档和资料, 并能灵活运用你的搜索引擎.
我们在群里提供了一些常见的文档, 它们应该足够应付你的作业了(不是本课, 而是整个大学).
熟悉它们, 边看边实现. 建议从 lshort 开始.

新手的一个常见的问题是, 不知道怎么开始工作. 或者, 很难独立完成一个 hello world 级别的工作.
这时候, 去找一两个其他人写的例子是一个好主义. 很多网站提供了这些例子:

\begin{enumerate}
\item \LaTeX 例子: https://www.overleaf.com/latex/examples
\item ctex 使用: http://www.ctex.org  
\item beamer 技巧: https://latex-beamer.com/faq/
\item 如何画矢量图: https://texample.net/tikz/examples/
\end{enumerate}

光看是不行的, 以上技能如果你想掌握, 必须切实去实际操作才行. 所以我们的第一天作业, 就是用 \LaTeX
撰写一篇基础的数学文章. 考虑到有同学是第一次接触 \LaTeX, 所以我会提供一个模板.
你只需在此基础上继续完成你的作业. 当然, 欢迎已经掌握 \LaTeX 的同学充分炫技. 

\bibliographystyle{plain}
\bibliography{crazyfish.bib}

\end{document}
