\documentclass{ctexbook}  
\usepackage{fontspec}
\usepackage{ctex}  
\usepackage{amsmath}
\usepackage{amsthm}
\usepackage{amssymb}
\usepackage{mathtools}
\usepackage{mathrsfs}
\usepackage{tikz}
\usepackage{pgfplots}
\usepackage[normalem]{ulem}  
\usepackage{subcaption}  
\usepackage{framed} 
\usepackage{enumitem} % 需要引入 enumitem 宏包
\usetikzlibrary{arrows,intersections}

%\setCJKmainfont{Source Han Serif SC} % 将主字体设为思源宋体,需要电脑安装此字体  
  
% \newenvironment{remark}  
%   {{\par\bf 注记: } \fangsong }  
%   {\par}

\newenvironment{Example}  
  {{\par\noindent\bf 例子: } \kaishu }  
  {\par}

\newfontfamily\yuanti{cwTeXYen} 

% 创建一个特定的定理样式和计数器  
\newtheorem{theorem}{定理}  
%\newtheorem{example}{例}

\newtheorem{definition}[theorem]{定义} % 定义 'definition' 环境    
% 创建其他定理样式并使用上述“theorem”的计数器  
\newtheorem{lemma}[theorem]{引理}  
\newtheorem{corollary}[theorem]{推论}  
\newtheorem{proposition}[theorem]{命题}  
%\newtheorem*{proposition}{命题}  
\newtheorem*{theorem*}{定理}  
\newtheorem{example}[theorem]{例}
\newtheorem{note}[theorem]{注记}
\newtheorem{challenge}[theorem]{挑战}  
\newtheorem{question}[theorem]{问题}  
\newtheorem{remark}[theorem]{备注}  

\numberwithin{theorem}{chapter}  
%\numberwithin{equation}{section} 
%\numberwithin{example}{section} 
%\numberwithin{definition}{section} 

%\numberwithin{figure}{section} 

\makeatletter  
%\@addtoreset{example}{section}  
\makeatother

\newcommand{\PP}{\mathbb{P}} 
\newcommand{\EE}{\mathbb{E}} 
\newcommand{\V}[1]{\mathbf{#1}}
\newcommand{\RR}{\mathbb{R}}
\newcommand{\NN}{\mathbb{N}}
\newcommand{\QQ}{\mathbb{Q}}
\newcommand{\CC}{\mathbb{C}}
\newcommand{\ZZ}{\mathbb{Z}}

\newcommand{\SR}{\mathscr{R}}
\newcommand{\Sol}{\noindent {\bfseries 解}}

\title{数学软件}  
\author{Wang Heyu}  
\date{\today}  

\begin{document}  
\maketitle  

\chapter{环境设置}

数学家是最不依赖工具的群体之一。在人类文明史的大部分时期,数学家们只依靠纸笔和大脑来进行研究,如果说要有什么辅助神器,
那么也就是黑板和粉笔,最多把咖啡算上。以上这些设备和环境,至今仍然对数学家,哪怕是应用数学家也非常重要。这也是我们学院的咖啡这么优秀的原因之一。
但是随着计算机这个某种程度上说是数学家的工作成果之一,它的迅速发展,也开始成为数学家,特别是应用数学家的重要工具之一。这就是我们这个课程开设的原因。
本课程的主要目的是引导未来的数学家,也就是你们,获得一些计算机基本技能,以及一些数学软件的使用技巧。
这些技能和技巧,将会在你们的学习和研究中发挥重要作用。

但是时间到了 2022 年底,事情发生了进一步的变化。在那一年的 11 月,openai 公司发布了第一个 ``真正让人惊讶'' 的大预言模型:chat-GPT 3.5,
从那以后,大语言模型开始以前所未有的速度发展。到今天,我估计在座的每一位同学,都已经在某种程度上开始使用大语言模型,
可能帮你收集学习资料,可能帮你写作业,写年终小结,甚至写情书。

现在每个接触过大语言模型的人都承认,这个技术已经改变了我们的生活,工作和学习。而且势必会继续改变。那么我们更加关心的问题是,它会改变数学么?
会改变数学的学习方式或者工作方式么?回答是肯定的,但是多大程度呢?是否像 Will Bryk @WilliamBryk 说的那样,大语言模型会取代数学家?
甚至会在 700 天内解决数学中的所有问题?我个人觉得没有这个可能性,或者说最好我是对的。如果机器真的可以取代数学家,那么人类也就没有存在的必要了。
但是通过这一年多的学习和使用,我体会到大语言模型在数学学习和工作中确实能够实质性地提升效率,因此,我比较保守地估计,
大语言模型很快将改变大部分数学家的工作方式,大部分学生学习数学的方式,以及我们的教学方式。因此,我将大语言模型的使用也纳入我们这门课程当中,
并且贯彻始终。但是注意:
\begin{enumerate}
  \item 我并不是教你们使用大语言模型。在面对这个新事物,我们都是初学者。因此,我会向你们展示我的使用方式,仅供大家参考。
  大语言模型是非常个性化的工具,每一位同学都应该形成自己的工作习惯,并且欢迎将自己的心得体会分享给大家。
  \item 在一些核心事务上,我仍然是非常保守的。比如我个人建议,不要相信大语言模型给出的数学证明,除非你自己能够验证。
  不要直接使用大语言模型给出文献,你必须去验证这个文献确实存在,内容也正确。最后,我们鼓励大家使用大语言模型来完成作业和项目任务,
  但一个评判的标准是,尽量不要让我和助教发现你使用了大语言的客观证据。每一个疑点都是扣分因素,比如奇怪的语法错误,
  或者一篇只有平行世界才会出现的文献。一个极端的例子是如果作业中出现了直接的大语言模型输出提示词,那么这个作业将被直接判零分。
  \item 一个无法评判,但是至关重要的原则是,不要让大语言模型代替你在核心问题上的思考,对于我们而言,就是数学问题。我们可以和它讨论,
  可以部分采纳它的结果,但是一定要以你为主,不要成为机器的挂件。
\end{enumerate}

因为大语言模型的缘故,接下去会出现大量非常主观和个性的内容,因此有必要介绍一下我的个人背景。

正如前面所述,我们的课程名称其实有点过时,但我们的课程目的倒是恰好,这是一门数学院的实践环节课程。课程的考查方式是作业和项目,也就是说,
没有考试。我希望大家在课程中追求的不是 GPA,而是真正能提高你数学学习和工作的技能。

课程的主要内容如下:
\begin{itemize}
  \item AI 工具的使用:这里主要指大语言模型。我推荐的是 Deepseek, Qianwen 和 Kimi,这三个模型都是国产且免费。建议可以注册账号。
  我们浙大也有本地化的版本,大家可以去看一下 chat.zju.edu.cn / open.zju.edu.cn. (暂时不支持视觉模型,所以我还是使用千问)
  \item 搭建数学工作环境:我们需要一个 Linux 终端环境,我个人使用的是 wsl2 + debian。你可以选择其他的 Linux 发行版,或者 Mac OS。
  具体安装和设置方面的问题你可以询问大语言模型,这是大模型的一个重要应用。在你没有其他经验的前提下,建议你的环境配置和我或者助教一致。
  这样一些具体的问题也容易找我们解决。此外我们需要一个编辑器,我个人使用的是 vscode,你也可以选择其他编辑器,比如 vim 或者 emacs。  
  \item 培养现代化的数学工作习惯:\LaTeX 是数学社区的标准文档工具。也是大家必须掌握的。与此相关的,我们需要用 Bibtex 来进行文档管理。
  然后作为应用数学工作者,还需要掌握一些基本的编程技能来实现数值模拟和数据处理。这里我推荐 Python 作为主要编程语言。
  如果我们的代码是应用层次的,那么我们可能还需要 C++ 这样的基础编程语言。
  \item 常用数学软件介绍:除了上面已经提到过的各种软件,我们还会接触一些和数学研究相关的软件,比如 Matlab,Mathematica,Maple 等等。
  以及一些可能不被看作软件,但是在应用数学中是重要的计算库,比如 BLAS,LAPACK,Eigen 等等。还有比如和绘图和数据处理有关的软件。
  我们用到的时候再说。
  \item 综合上述技能完成一些数学建模和科研任务:我们需要有一些项目管理的概念的工具支持,比如 git,make, cmake 等等,
  以及一些相关的软件工程概念。
\end{itemize}

这门课的开始对象是数学院的本科生。因此我们在技能包括大语言模型的使用技能的讨论将集中在数学领域。非数学专业的同学,请根据实际情况慎重考虑。


% 这一点同样也影响了数学的学习和研究。在这个技术出现的头几个月,大多数数学家都是抱着一种 ``这个东西不会对我有什么影响'' 的态度。但是很快,至少我自己就发现,
% 这个东西能极大提高我的工作效率。比如尽管我理论上可以直接阅读英语,但通过它我能够以阅读母语的速度阅读几乎任何外文文献。

\begin{example}
  接下去我们直奔主题,通过一个实际工作案例来了解数学工作者的一个重要日常行为:阅读论文。
  这里我们选取 AI 的基础之一,随机梯度下降算法的开创性论文:
  Herbert Robbins \textup{\&} Sutton Monro 1951, A Stochastic Approximation Method\cite{robbins1951stochastic}。 
  接下去是否理解论文内容本身不是重点,请大家把重点放在我是如何阅读的具体细节上。这篇论文中的理论内容可能需要二年级以上的数学背景才能完全理解。
  但不排除通过大语言模型的帮助,大一的同学能够理解至少一部分内容。

  \begin{enumerate}
    \item 其实这里跳过了一个重要的过程,就是如何选择需要阅读的论文。大家可能已经注意到学术论文的产量是非常惊人的。
    而我们研究的第一出发点一般是兴趣。所以最常见的情况,是从一个或若干个关键词出发,去寻找相关的文献。在 23 年之前,
    我们基本上需要通过 Google Scholar 或者百度学术这样的搜索引擎来找到文献。但现在,通过大语言模型往往效率更高。
    如果你问了正确的问题,比如:“请推荐随机梯度下降法的关键论文。” 
    一般大语言模型只能给出文献索引,你需要自己去找到这篇论文,比如我们的图书馆。当然你也可以问大模型:“哪里可以下载到这篇论文?” 
    但这样的问题一般不会得到太正面的回答。要学会结合使用已有的资源。此外要注意,大语言模型有时会瞎编根本不存在的文献,
    所以提供的文献必须要验证。对待一篇论文的一般性态度是:我们有没有必要读,有没有必要精读?特别是我们在日常工作中,
    往往是根据关键字得到文献列表,
    如何筛选出我们真正需要的文献,是科研和学习的重要技能。一个简单的办法是在不下载原文的情况下,直接问大模型:“请给出这篇文章的主要内容。”
    这有助于我们做初步的筛选。
    \item 对于初步筛选的论文,我们可能需要做的选择是是否进一步精读。在这个阶段,我们应该已经获得了下载的原文。所以我们可以上传这篇论文,
    然后请求大模型做分析和总结。这里由于大模型理论上并不会记忆你上传的数据,所以通常情况下我们不用考虑泄密的问题。当然机密文件除外。
    一般比如说你尚未投稿的论文,通过这种和大模型对话的方式是不会泄露的。我们可以上传之后问大模型:“请分析这篇文章的主要内容和结论。”
    我个人对于一篇论文,有三种程度的阅读:1. 只看摘要和结论; 2. 全文翻译;3. 全文翻译并做精细的笔记。
    第一个层次还是在泛读阶段;如果进入第二个层次,说明我想了解这篇论文的主要内容和贡献,此时大模型能够提供接近母语的阅读效率。
    而第三个层次说明我真的想掌握这篇论文提出的方法和结果。在以往,这个过程意味着手抄和笔记。但现在通过大模型,
    我更加倾向于利用大模型的视觉模型,也就是把每一段论文的照片发给它,让它翻译。此时大模型会集中注意力于字面翻译,
    你也可以主动要求大模型尽量给出专业的翻译,不要评论,不要扩展讨论,如果有看不清的地方,用上下文推断,
    最后再以 latex 格式输出。这样你就能得到一个非常好的格式化翻译文本,以便精读。我们需要逐句检查翻译的准确性,和推理的正确性,
    在这个过程中,我们会像手抄一样仔细检查每一个符号,从而获得接近手抄的理解效果。当然这一点有一定的个人因素。
    \item 到这里为止,大模型只是起到一些辅助的作用。对于我们关心的核心工作,比如论文中的主要定理的证明等等,仍然需要我们自己去理解。
    但是对于一些具有深度学习的大模型,比如 GPT O1/O3,Deepseek R1,Qianwen qvq 和 Kimi K1,它们在数学问题上的表现已经非常出色,
    甚至可以提供一些有益的讨论。 尽管这里我选取的例子是阅读论文。但是对于一门专业课程,不论是课本还是参考书,我们都可以用类似的方法来阅读。
    比如我在作业包中分别提供了分析和代数的经典教材,陶哲轩的《分析学 I》\cite{Tao2016AnalysisI} 
    和 Gilbert Strang 的《线性代数导论》\cite{Strang2016LinearAlgebra}。除了完成作业,如果有兴趣的话你完全可以继续精读这两本数,特别是第一本。
    你甚至可以针对这两本书建立智能体,然后和它们进行对话。(浙大先生似乎支持这个操作,但我还来不及实验。在通义千问中试一下。)
    \item 作为应用数学的研究工作,涉及数值模拟和数据处理几乎是必然的。根据目前的态势,我推荐 Python 作为主要编程语言。而在具体环境的选择上,
    我选择的是 Anaconda。这个环境非常适合数学工作者,因为它集成了大量的数学和数据处理库,而且支持交互式计算。作为一个演示,
    我们可以要求阅读完论文的大模型提供一个数值算例。或者提供一个数值算例的思路,让大模型来编程实现。
    \item 如果我们不是完全的翻译和精读,我们也可以在讨论完毕之后,要求大模型提供一个总结性报告,甚至是幻灯片。以供我们后续学习。
    所以有了大模型之后,文档的产生效率大大提升,随之而来的问题是如何管理好我们的文档。除了上面看到的 Bibtex,我们还需要学会使用一些文档和项目管理工具。
    
  \end{enumerate}
\end{example}

% 到这里我们还可以说大模型只是做了一些文书工作。但是当具备深度学习的 GPT o1,
% Deepseek R1,Qianwen qvq和 Kimi K1 等深度思考模型出现之后,甚至我们都可以和它们讨论专业数学问题了。就像我的一位同事根据亲身经历的反馈,
% 他把自己正在探索的开放问题提交给了深度思考模型,
% 模型仔细思考后给出了一大堆似是而非的结果,非常像一位没有准备的学生在考试中的表现。但是,其中它提出的一种证明,尽管一看就是错误的,
% 但提示了他两个看上去完全无关的理论之间的联系,
% 这给他了一个全新的方向性启示,然后按着这个思路走下去,他最后完成了这个工作。我自己也有类似的感受,和深度思考模型讨论,
% 非常想和一位基础扎实,但思维脱跳的学生讨论。
% 而且这个学生的推理能力提升极快,目前这几个月给我的感觉,已经超过了硕士生的平均水平[doge]

% 再比如它确实能够完善我的代码,
% 让我在写代码的时候更加专注于算法本身,而不是一些语法细节,同时它让我的代码具有完善的注释和文档,以及严谨的风格,优美的缩进等等。
% 当让它也能帮我快速地完成一些非常枯燥的文书工作,比如工作汇报,年度总结等等。最后,尽管我已经过了要写情书的年纪,但它确实能极大提升我的 email 和推荐信的书写质量,
% 特别是英语场合。可以说,目前在我的工作、学习和生活的很多场景,我都已经有意无意地大量使用大语言模型,这里也包括了我的专业工作。而且通过和我的同事的交流,
% 我发现这并不是我的个人感觉,事实上几乎我的所有同事,都已经在自己的研究工作中或多或少地使用这一重要工具,这远比我们当初预计的影响要大得多。特别是

% 所以几乎可以肯定地说,未来的数学家,哪怕是纯粹数学家,也很难不使用大语言模型来提升他的工作效率。所以从这个学期开始,我们进一步对我们的课程进行改革,
% 我们将更多地引入大语言模型相关的内容,并且在整个课程中,鼓励大家去主动使用大语言模型,来提升自己的学习和工作效率。所以我们先来设置或对齐一下我们的工作环境。

% \section{大语言模型}
% 大语言模型(Large Language Model,LLM)是一种基于深度学习技术构建的人工智能模型,专门用于处理和生成自然语言文本。这个定义本身是 Kimi K1.5(深度思考版本)
% 在联网的情况下给出的。与此同时 Deepseek R1 的回答是:服务器繁忙,请稍后再试。这也是我推荐和个人使用最多的两个国产大语言模型。我个人的感觉是这两个模型都很好,
% Qianwen 和 Kimi 更加偏文科生一些,能处理较长的文本,翻译和文书写作更好一些,同时 Kimi 的 K1 也有一定的推理能力;而 Deepseek R1 则很像经典印象的理科生,
% 有很强的逻辑推理能力,
% 文字表述方面偏简洁,但如果你给了很明确的指示,它也能做的很好。当然目前使用 Deepseek 的主要困难在于服务器反应不够及时,这也是我同时推荐 Qianwen/Kimi 的原因之一。
% 不管怎么说,这些模型都足够我们这门课程使用了。此外,这些模型都是免费的,强烈建议大家如果还没有账号,都去注册一个。

我们这里先简单地讨论一下大语言模型的本质,因为这有助于我们进一步合理使用它们。一个训练完毕的大语言模型,是一个具有极高维数(一般有大约 7000 亿个参数),
并带有复杂网络结构的知识图谱。它当中几乎覆盖了我们人类目前所有的知识,但它并不是一个经典意义上的数据库。尽管训练它时,人类输入了大量的文献,
但这些文献本身并不在这个知识图谱中,这个知识图谱中记录的只是人类在这些文献的引导下回答相关问题时组织对话的频率(我们不用概率,因为尽管参数量极大,
但毕竟还是有限个)所以从这个意义上说,它更像是“熟读唐诗三百首,不会作诗也会吟”。它将人类的知识库,抽象成一个超高维的流形曲面。这个曲面只存在在概率空间中,
我们不可能精确地绘制它,我们只是通过随机采样得到的参数去逼近它,这个过程被称为“训练”。等训练完成后,我们的全部随机参数就可以看作是这个人类知识曲面的一个逼近。
而推理过程,就是根据提问形成一个局部曲面片,然后用概率模型去寻找最一致的语言序列最为后续回答。因为整个推理过程也是随机的,因此即便相同的问题,我们也有可能得到不同的回答。

更深入的算法细节,我们在以后的专题再讨论。
而这里我们就先非常粗浅地说,大语言模型本质上是一个文科生。它对问题的回答内容,匹配的是人类文献中类似回答的频率,本质上它的思维能力仍然是有限的,
同时它并不会严格意义上的逻辑推理。这一点在早期的模型中表现的非常明显。它在回答简单的数学问题的时候,比如整数的加法,它都可能回答错误,
因为它只是以一个符合人类回答频率的方式随机给出一个答案。这也说明平均意义上的人类,事实上也并不是很擅长数学。
深度思考模型的出现其实就是为了解决这个问题。深度学习网络开始学习真正的人类逻辑推理,并用逻辑推理去检查大语言模型的输出内容是否有逻辑上的冲突。
目前这方面的突出成果就是大语言模型在数学问题上的表现越来越出色,比如我们问它一个有一定难度的微积分甚至数学分析的证明,它也有一定可能给出正确的答案。


\bibliography{../mathsoft.bib}
\bibliographystyle{plain}


\end{document}  

