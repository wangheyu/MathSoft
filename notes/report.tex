\documentclass{article}
\usepackage[utf8]{inputenc}
\usepackage{ctex} % 加载 ctex 宏包以支持中文
\usepackage{amsmath}
\usepackage{amsfonts}
\usepackage{amssymb}
\usepackage{graphicx}

\title{基于 Robbins-Monro 方法的随机逼近数值算例设计与实现}
\author{}
\date{}

\begin{document}

\maketitle

\section{引言}
Robbins-Monro 方法是一种经典的随机逼近算法,最早由 Herbert Robbins 和 Sutton Monro 在 1951 年提出。该方法通过迭代更新的方式,利用带噪声的观测数据逐步逼近目标值。尽管原论文中并未包含具体的数值算例,但这一方法在统计学、机器学习和优化领域具有广泛的应用。

本文旨在通过一个简单的数值算例,展示如何使用 Robbins-Monro 方法来估计未知参数,并验证其有效性。

\section{方法介绍}
\subsection{目标问题}
假设我们有一个未知的参数 $\theta^*$,并且我们只能通过带噪声的梯度观测来接近它。目标是使用 Robbins-Monro 方法找到 $\theta^*$ 的估计值。

目标函数定义为:
$$
f(\theta) = (\theta - \theta^*)^2
$$

我们无法直接观测到 $f(\theta)$,而是通过带噪声的梯度观测 $Y_n$ 来逼近:
$$
Y_n = 2(\theta_n - \theta^*) + \epsilon_n
$$
其中,$\epsilon_n$ 是均值为零的随机噪声(例如正态分布 $N(0, 1)$)。

\subsection{迭代公式}
Robbins-Monro 方法的核心迭代公式为:
$$
\theta_{n+1} = \theta_n - a_n Y_n
$$
其中:
\begin{itemize}
    \item $\theta_n$ 是第 $n$ 次迭代的估计值。
    \item $a_n > 0$ 是步长序列,满足以下条件:
          \begin{enumerate}
              \item $\sum_{n=1}^\infty a_n = \infty$
              \item $\sum_{n=1}^\infty a_n^2 < \infty$
          \end{enumerate}
\end{itemize}

为了简单起见,可以选择 $a_n = \frac{1}{n}$。

\section{数值实验}
\subsection{参数设置}
\begin{itemize}
    \item 真实参数值:$\theta^* = 3$
    \item 初始估计值:$\theta_0 = 0$
    \item 随机噪声:$\epsilon_n \sim N(0, 1)$
    \item 步长序列:$a_n = \frac{1}{n}$
    \item 迭代次数:1000 次
\end{itemize}

\subsection{实现代码}
以下是基于 Python 的实现代码:
\begin{verbatim}
import numpy as np
import matplotlib.pyplot as plt

# 参数设置
theta_star = 3  # 真实参数值
num_iterations = 1000  # 迭代次数
theta = 0  # 初始估计值

# 存储每次迭代的结果
theta_history = [theta]

# 迭代过程
for n in range(1, num_iterations + 1):
    # 计算带噪声的梯度观测
    epsilon = np.random.normal(0, 1)  # 随机噪声
    Y_n = 2 * (theta - theta_star) + epsilon
    
    # 更新估计值
    a_n = 1 / n  # 步长
    theta = theta - a_n * Y_n
    
    # 记录当前估计值
    theta_history.append(theta)

# 输出最终估计值
print("最终估计值:", theta)

# 可视化结果
plt.plot(theta_history)
plt.axhline(y=theta_star, color='r', linestyle='--', label='真实值')
plt.xlabel('迭代次数')
plt.ylabel('估计值')
plt.title('Robbins-Monro 方法估计过程')
plt.legend()
plt.show()
\end{verbatim}

\section{结果分析}
\subsection{收敛性}
通过运行上述代码,我们可以观察到估计值 $\theta_n$ 随着迭代次数增加逐渐收敛到真实值 $\theta^* = 3$。尽管由于观测中存在随机噪声 $\epsilon_n$,估计值可能会有一定的波动,但步长 $a_n$ 的递减特性确保了最终的收敛。

\subsection{噪声影响}
随机噪声的存在使得估计值在初期可能存在较大的偏差,但随着迭代次数的增加,噪声的影响逐渐减弱,估计值趋于稳定。

\section{总结}
本文通过一个简单的数值算例,展示了 Robbins-Monro 方法的基本原理及其应用。实验结果表明,该方法能够有效逼近目标值,即使在存在噪声的情况下也能保证良好的收敛性。此外,步长序列的选择对算法的性能至关重要,合理的设计可以显著提高收敛速度和精度。

未来的研究方向可以包括:
\begin{itemize}
    \item 探讨不同步长序列对收敛性的影响;
    \item 将该方法应用于更复杂的优化问题或实际工程场景中。
\end{itemize}

\end{document}